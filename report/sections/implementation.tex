\chapter{Implementation}
  This section will explore various aspects of the implementation and choices made during the development
  of Echo. Any functions or concepts introduced or modified in this section will be expressed in terms
  of the Fran semantics specified in section~\ref{sec:semantics} where possible. The source code for
  the implemented framework can be viewed on Echo's source repository \cite{Stott}.
  
  \section{Scala}
    Scala is a multi-paradigm programming language developed to run on the Java Virtual Machine \cite{Odersky2004}. Scala 
    programs are
    generally structured in an object oriented programming style like Java but it also incorporates many features of
    functional languages such as closures, high order functions, lazy evaluation, immutable state and currying.

    Scala also possesses an advanced, powerful and feature rich static type system. This provides large
    advantages when implementing FRP. Firstly, Scala supports generic classes and it can be quite easily seen
    that both Behaviours and Events will rely on this feature for type safety. Scala additionally provides
    a rather unique type conversion functionality that allows types to be converted on the fly. This provides
    incredibly expressive and powerful ways to embed FRP into Scala as will be demonstrated later.
    
    One slight problem with implementing FRP in Scala was that, unlike Haskell, it is an `impure'
    language. In terms of functional programming, `purity' refers to a languages ability to write functions
    thats results are affected by some external estate or in turn effect some external state. For instance,
    this means that a function that defines a Behaviour could be impure and could therefore cause side effects
    whenever the Behaviour was evaluated. There has been some work into pure function verification in Scala but it has not 
    materialised into an  
    implementation \cite{Nordenberg}. Due to this, we assume that Behaviours are defined using pure functions in Echo.
    
  \section{Object Orientation}
    Behaviours and Events are implemented in
    Echo using Scala's object oriented definitions. For instance, a Behaviour[T] is simply
    implemented as a generic class with a Time =$>$ T function parameter. An example Behaviour
    could be instantiated as follows:

\begin{verbatim}
val beh = new Behaviour[Double](time => sin(time))
\end{verbatim}  

    A discussion of Event's implementation will feature later.
    
    It would be possible to implement the original FRP operations as standalone functions. However, so
    as to conform to Scala's object oriented style the operations will be attached to their respective
    classes. For instance, the \emph{merge} operation will be used in the following way in Echo:
    
\begin{verbatim}
val merged = event1.merge(event2)
\end{verbatim}

  This allows the FRP code to fit more naturally into Scala. We will also replace the \emph{switcher} function
  with a type that extends the Behaviour type:
  
\begin{verbatim}
val switch = new Switcher(iniBeh, behEvent)
\end{verbatim}

  We also define a type for the \emph{stepper} function in a similar manner.
  
  \section{Extendable Events}
    FRP Events provide a very useful abstraction for external stimuli such as I/O and user input. Of course
    to allow for Events to interact with this input they must be allowed to actually `occur'.
    It is noticeable that the original denotational semantics \cite{Elliott1997} do not explore how to achieve
    this and it is assumed that this is because the actual occurring of Events is really just an implementation
    issue. As discussed earlier in section~\ref{sec:semantics}, previous implementations of Events in Scala add an operation to Event's
    public API to allow it to occur. Here we will show an alternative approach.
    
    When observing how Events are composed (using \emph{merge} and \emph{map} operations) it can be seen that
    only a subset of Events actually `occur' while others are simply views and combinations of these `source' Events.
    This is modelled directly in the implementation of Echo. FRP Events are represented by two traits \footnote{Traits are abstract
    classes that can possess concrete members. A class can extend from multiple Traits (like Java's interfaces)}: 
    Event[T] and EventSource[T].
    Event has concrete implementations of \emph{map} and \emph{merge} operations that
    return concrete Event[T] instantiations. EventSource[T] extends
    from Event[T] and additionally has a protected access concrete function \emph{occur}:
    
\begin{verbatim}
def occur(value : T) : Unit
\end{verbatim}  

    This function causes the EventSource[T] to `occur' with the given value (with the time the call was made). This allows 
    programmers
    to easily build FRP Event abstractions for any form of input without exposing the \emph{occur} action externally.
    As an example, Here is a simple implementation for an Event that occurs once with the value `5' when it 
    is created:
    
\begin{verbatim}
class Five extends EventSource[Int] {
  occur(5)
}
\end{verbatim}

  \section{Companion Instantiation}
    One ugly aspect of Scala is its inheritance of the \emph{new} keyword from Java as the standard
    method of calling class constructors.
    We can however avoid using this in an attempt to lower the density of code written with Echo. For instance, Behaviours and other objects can be instantiated in the following way:

\begin{verbatim}
val beh = Behaviour(time => 5)
\end{verbatim}

    This is accomplishable using a combination of two Scala features: Companion objects and \emph{apply} functions
    \cite{Typesafe}. Using these features we can implement a static function that is executed whenever function
    arguments are applied to the class name. We can then simply call the the class' constructor from this. An implementation
    that allows for the above example is as follows:

\begin{verbatim}
object Behaviour {
  def apply[T](func : Time => T) = {
    new Behaviour(func)
  }
}
\end{verbatim}    
    
  \section{Flat Time}
    Time in Echo is implemented as a flat value type defined exactly as it is in 
    section~\ref{sec:semantics} of this report (as an alias of Double). The internal implementation does take advantage
    of general assumptions about Time. For instance, the \emph{merge} function for Events takes advantage
    of the fact that an Event without occurrences still has a has a lower bound of when it can occur (again described
    in section~\ref{sec:semantics}). 
    
    So we can refer the `present' we define a function for calculating this in Echo:
    
\begin{verbatim}
def now() : Time = System.nanoTime() - startTime
\end{verbatim}      

    Here \emph{startTime} refers to the time the current Echo program started executing. It should be noted that \emph{nanoTime} (returning current time in
    nano seconds) is the most accurate value available for the current system time in Scala. But, due to the
    implementation of the Java Virtual Machine \emph{now} will start to return incorrect results 
    after a program has been running for around 292 years due to an overflow in the calculation (thought to be an allowable
    problem).

  \section{Lifting Constants}
    As mentioned earlier, one of the most powerful features of Scala is it's ability to implicitly convert between
    types at run time. This is facilitated by the \emph{implicit} key word allowing a function to be defined
    that is automatically executed when a corresponding type mismatch occurs. This functionality is used in Echo to allow 
    static values to be converted to Behaviours implicitly and can be implemented in the following way:

\begin{verbatim}
implicit def liftConst[T](value : T) : Behaviour[T] = {
  Behaviour(time => value)
}
\end{verbatim}              
    
    This allows us to express simple `constant' Behaviours as values:

\begin{verbatim}
val beh : Behaviour[Int] = 0
val switcher = Switcher(false, eventBeh)
\end{verbatim}       

    It can be seen that this allows the programmer to switch between static and more complex
    Behaviours in a very naturalistic manor.
    
    Another advantage of lifting constants in this way is that we can guarantee that Behaviours
    returned from \emph{liftConst} will have the same value at any point in time. We exploit this by returning a Behaviour
    that simply returns a value when evaluated rather than evaluating a function with respect to time. These optimised Behaviours are still externally identical to normal Behaviours however so no functionality is lost.
    
    \section{Lifting Functions}
    Lifting of operators and
    functions presents a few problems. This is because functions in Scala (as discussed earlier)
    are usually attached to classes. For instance, there is no `+' operator in Scala: the numeric
    types have a \emph{+} function defined in their class declarations (how Scala actually performs additions
    is a topic for discussion elsewhere). This means that we cannot `lift' many of the operations we
    would want to. To solve this problem we can use a series of map functions. We can define these
    here as part of Behaviour[T]:

\begin{verbatim}
def map[U](func : T => U) : Behaviour[U]
at(beh.map(func), t) = func(at(beh, t))

def map2[U, V](beh1 : U)(func : (T, U) => V) : Behaviour[V]
at(beh.map2(beh1)(func), t) = func(at(beh, t), at(beh1, t))
\end{verbatim}        

    These functions allow Behaviours to be transformed and combined using functions
    normally applied to static values.

  \section{Time Travel is Dangerous}
    \label{sec:time}
    Relating back to the discussion of interactive Behaviours in section~\ref{sec:subtle} it can be seen
    that there are problems when plugging FRP into the real world: the future is uncertain and
    arbitrary access to the past requires an unbounded level of memory. After much deliberation
    over different compromises it was decided that Echo would follow a set of rules:
    
    \begin{itemize}
      \item If a Behaviour was last evaluated at time \emph{u} then it can only be evaluated at
      a time \emph{t} if \emph{t} $\geq$ \emph{u}.
      \item An Event holds the last time it occurred and the time up to which we know it hasn't occurred
      again (the time it is `valid' until).
      \item An Event is valid up until the time equivalent to `now' (the actual current time).
      \item If a Behaviour depends on some Event then we can only evaluate the Behaviour up to
      the time the Event is valid. We additionally can only evaluate it at a time greater than
      or equal to the Event's last occurrence.
    \end{itemize}
    
    These rules restrict Behaviours to what we will refer to as `real-time' evaluation as the times
    we evaluate a Behaviour at become monotonic in nature. Additionally, as Event occurrences will
    be happening in real time we will see that the lower bound that we can evaluate a Behaviour at will
    constantly be pushed up to the present (leaving only `now' as a possible evaluation time). Although
    other times will be available for evaluation at times (there will often be a gap between `now' and
    the last Event occurrence) it would be monotonous to expose these bounds to the programmer so
    in Echo explicit evaluation of a Behaviour with the
    \emph{at} operation is entirely removed from the public API and replaced with a real-time evaluation function:

\begin{verbatim}
def eval() : T
beh.eval() = at(beh, now())
\end{verbatim}  

    This restriction may seem like a significant loss of expressivity. However as we wish to be able to treat
    Behaviours and Events as first class values we would actually like to avoid ever explicitly evaluating
    them: this would ideally be carried out by abstractions for graphics or other kinds of output. For instance
    if we wanted to set some colour in a user interface equal to a Behaviour then explicitly evaluating the Behaviour
    periodically and setting the colour using a static value would defeat the point of using FRP. We will see in later
    discussion that we can easily create abstractions to hide this type of logic. Additionally, this
    choice to not expose Behaviour evaluation to the programmer is also seen in the frameworks
    discussed in chapter 3.
    
    It should also be noted that restricting evaluation times for Behaviours forces us
    to remove the \emph{timeTransform} operation for Behaviours as this function allows
    a programmer to create a Behaviour that at a time \emph{t} would be evaluated in respect
    to a time less than or greater than \emph{t}. The restriction also means we are unable to implement
    the \emph{predicate} operation included in Fran. This is because it would either require evaluating Behaviours
    at future times (not allowed) or evaluating past intervals periodically. The former would provide a functional
    implementation but it would mean that Events would be created in the past. This would cause 
    inaccurate states in the internal FRP representation and so is not an acceptable compromise.
    
    These rules being the case allows Echo to circumvent the semantic problems of evaluating the future
    and also prevents it from having to store an Event's entire past. With only these restrictions we may still have to
    buffer some occurrences as we would not want swap in new occurrences that happen while an Event is being evaluated.
    
    It should be noted that restricting evaluation times does not alter the original semantic model of FRP as our description
    of Behaviours and Events still holds: we have merely restricted the values that can be used in the evaluation of the
    \emph{at} function.
    
    \section{Setup Phase}
      There was one further problem presented due to the restriction on Behaviour evaluation. This was concerned with
      implementing the \emph{snapshot} operation. To illustrate the problem, imagine we execute the following code
      at time \emph{t}:

\begin{verbatim}
val ev = beh.snapshot(someEvent)
\end{verbatim}

      This is fine if \emph{someEvent} has not occurred at \emph{t}. If it has occurred however at some time
      before \emph{t} then in whatever way this is implemented we are forced to evaluate \emph{beh}
      at a time in the past. If \emph{beh} depends on any Events this time might be before the Event's
      last occurrence and so we would effectively be returning an incorrect result. Dealing with this
      presented an interesting problem. It was decided that FRP
      code (FRP operations on FRP types) could only be performed in an initial setup phase of a program.
      This setup phase would happen `instantly' as the internal notion of time is frozen and Event
      occurrences are not processed by the framework during this phase. As Scala programs are defined inside a singleton
      object it was decided that a user would simply be able to extend a class with this object and then
      write their FRP code in a \emph{setup} function:

\begin{verbatim}
object App extends EchoApp {
  def setup(args : Array[String]) {
    // FRP code here
  }
}
\end{verbatim}
  
    This allows the FRP world to be setup in one distinct stage. It should be noted that
    any functions \emph{setup} calls would also be able to run FRP code. Only FRP code that runs
    after the \emph{setup} function has finished executing will cause an error.
    
    Behaviour and Event instantiation is not blocked after the setup phase however. This is
    to allow for Event[Behaviour[T]] or Event[Event[T]] objects to \emph{occur}
    dynamically.
    
  \section{Atomic Input \& Output}

     By this point it is noticeable that the FRP world that Echo builds only really has one form of input and one
     form of output action: \emph{occur} and \emph{eval} respectively. Conceptually we think of both of these
     actions happening instantly. For instance, when a Behaviour is sampled it will need to do some evaluation
     and during that evaluation nothing in the FRP world should change. This suggests that the \emph{occur}
     and \emph{eval} actions should both be atomic with respect to this world so an Event cannot occur during
     a Behaviour evaluation and vice versa. 

     This concept is implemented in Echo using a locking object for the FRP world. If a Behaviour is evaluated it
     must first acquire this lock and the same goes for the Event \emph{occur} action. This implementation
     also provides a significant advantage: Event occurrence times and evaluation times now form one
     monotonically increasing timed chain \footnote{We can actually model evaluation and occurrences of an Event as an 
     Event!} of actions. This fact means that we only ever need to store one occurrence for each Event: if an Event occurs
     at time \emph{t} it will not be evaluated until a time greater than or equal to \emph{t}. This allows
     Events to only use a bounded level of memory. 

     It could be argued that this implementation is problematic as there may be some lag between a real world
     occurrence and it being processed by the FRP system in Echo. For instance, if a user clicks on a button
     while some network Event is occurring the button click will have to wait to `occur' in the FRP system.
     However, if these actions weren't atomic there would be the possibility of latency inside the FRP
     system itself. Taking the example above, imagine the two real world occurrences are delivered on two
     separate threads and that the button click arrives at time \emph{t} and the network occurrence
     arrives later at time \emph{u}. In this
     situation our FRP system's accuracy is at the mercy of thread scheduling as the button click
     could be part of the way through the \emph{occur} action and then be scheduled out to allow the network
     occurrence to happen. If these Events had been merged we would effectively skip a state:
     we would never be able to observe the occurrence at \emph{t}. Atomic input and output allows us
     to avoid internal inaccuracies such as these and so is acceptable despite any slight lag it might
     produce with processing external input.
  
  \section{Constant Events}
    It useful to be able to define an Event that occurs only once. This is provided in Echo
    via the Event companion apply function:

\begin{verbatim}
val const = Event("Once")
\end{verbatim}    

    This would create an Event with a single occurrence of the value ``Once'' at time \emph{0} as
    it occurs at the end of the setup phase. Due to the occurrence time these Events can only created
    within the \emph{setup} function described earlier.
    
    Although it may not be as useful it should also be possible to instantiate a never occurring
    Event. This can be achieved in a similar manner:

\begin{verbatim}
val empty = Event()
\end{verbatim}
     
  \section{Function Renaming}
    Although a small change it should also be noted that one of the functions originally defined
    has been renamed for the Echo framework - \emph{snapshot} was renamed to \emph{sample}. This was done as
    it was felt it added more clarity to the function name and was more appropriate given its use.
    
  \section{Function Additions}
    When considering different applications of the framework and also in an attempt to
    tackle the changes in expressiveness between Haskell and Scala some operations
    were added to the FRP contained in Echo.
        
    In Hudak and Elliott's original paper an example is constructed to create a recursively defined Behaviour
    that toggles back and forth between two given Behaviours whenever an Event occurs. This
    example took advantage of Haskell's lazy evaluation. While Scala is able to use lazy evaluation
    it is not particularly intuitive and requires some rather awkward syntax. To tackle this
    a \emph{toggle} function was added to Echo to provide the functionality shown in Elliott and Hudak example out of the box.
    This is defined as a member of Behaviour[T]:
    
\begin{verbatim}
def toggle(ev : Event[T], beh1 : Behaviour[T]) : Behaviour[T]
\end{verbatim}        
    
    It was felt that another common use of combining Behaviours and Events would be
    to switch from one Behaviour to another when an Event occurs (but not back again). Again, this
    function is defined as a member of Behaviour[T]:

\begin{verbatim}
def until(ev : Event[T], beh1 : Behaviour[T]) : Behaviour[T]
at(beh.until(ev, beh1), t) = at(new Switcher(beh, 
  event.map((t,v) => beh1)), t)
\end{verbatim}        
    
    It is also useful to be able to to filter Events to produce a new one that only
    contains occurrences matching a predicate. This allows a Behaviour to only react
    to some occurrences from an Event. This is defined as a member of Event[T]:

\begin{verbatim}
def filter(predicate : T => Boolean) : Event[T]
occs(event.filter(predicate)) = occs(event).filter(predicate)
\end{verbatim}  

  \section{Behaviour Caching}
    \label{sec:caching}
    Behaviours possess quite a major inefficiency in the form of redundant evaluation:

\begin{verbatim}
at(beh, t)
at(beh, t)
\end{verbatim}  

    Here \emph{beh} is evaluated at \emph{t} twice. It can be seen that we could simply cache the last
    computed value for each Behaviour so we could return this value instantly if the Behaviour was sampled at that time 
    again.
    We take this approach internally in Echo so as to avoid redundantly recomputing values. 
  
  \section{Test Driven Development}
    Echo was mainly developed using Test Driven Development \cite{Cunningham}. This methodology is based around the idea
    of specifying some program feature with a test and then writing code to pass that test. The hope
    is that doing so will not just ensure testing is thorough (as all features will require tests) but
    that solutions are as simple as possible. TDD also encourages
    refactoring after each test/test passed phase.

    Test implementation for Echo was carried out using the Scala framework `Specs' \cite{EricTorreborre}. This framework
    encompasses a DSL that allows tests to easily be defined. For instance, this code tests the implicit lifting
    of constants feature in Echo:

\begin{verbatim}
"Echo" should {
  "allow values to lifted to constant Behaviours" in {
    val beh: Behaviour[Int] = 5
    beh.eval() mustEqual 5
  }
}
\end{verbatim}

    Carrying out testing such as this was imperative to the project goals as it was very important to be able
    to verify that code did not only compile and run but that implementations matched the original denotational 
    semantics. In other words, developing Echo with TDD allowed for verification that the implementation
    was correct throughout the development process.
    
  \section{Documentation}
    Of course for any programming language framework to be useful it must be documented for both potential users
    and developers who may want to expand the framework itself. 
    
    For user programmers two pieces of documentation
    are provided. The first of these is a specification for the core FRP package (Behaviour, Event, Stepper and Switcher
    types) that is provided in Appendix C of this report. Furthermore a tutorial on how to
    create Echo programs is provided on Echo's website \cite{Stotta}.
    
    Anyone wanting to expand Echo in any way is also provided with two pieces of documentation. Firstly, a Readme file exists
    in Echo's source that details how to compile the project, set the source up in various IDE's and describes the package
    layout for the code. There are also comments throughout the source that detail different aspects of the 
    implementation allowing programmer to gain an insight into the current implementation.