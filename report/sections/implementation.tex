\chapter{Implementation}
  As stated earlier the goal of this project was to implement a framework for utilising FRP in the Scala. 
  This framework was dubbed `echo'. This section will explore the motivations and problems ahead of
  and also the choices made and results rendered during the implementation of \emph{echo}.
  
  \section{Design Choices}
  Throughout its history FRP implementations have generally been carried out in Haskell. Due to the movement
  in setting and also in an effort to create a efficient implementation without sacrificing FRP's expressive power
  several design choices were made for echo. They will be explored here.
      
    \subsection{Object Orientation}
      It would be possible to implement the original FRP operations as standalone functions. However, so
      as to conform to Scala Object Oriented style the operations will be attached to their respective
      classes. For instance, the `merge' operation will be used in the following way in echo:
      
\begin{verbatim}
val merged = event1.merge(event2)
\end{verbatim}

    This allows the FRP code to fit more naturally into Scala. We will also replace the `switcher' function
    with a type that extends Behaviour rather than a function:
    
\begin{verbatim}
val switch = new Switcher(iniBeh, behEvent)
\end{verbatim}

    \subsection{Implicit Lifting of Constants}
      As mentioned earlier, one of the most powerful features of Scala is it's ability to convert between
      types at run time. This is facilitated by the \emph{implicit} key word allowing a function to be defined
      that is automatically when a corresponding type mismatch occurs. This will be used in echo to allows 
      static values to be converted to Behaviours \emph{implicitly}.. This is implemented in the following way:  

\begin{verbatim}
implicit def liftConst[T](value : T) : Behaviour[T] = {
  new Behaviour(time => value)
}
\end{verbatim}              
      
      This would allow the following for instance:

\begin{verbatim}
val beh : Behaviour[Int] = 0
val switcher = new Switcher(false, eventBeh)
\end{verbatim}       

      It can immediately be seen that this allows the programmer to switch between static and more complex
      Behaviours in a very naturalistic manor.
    
    \subsection{Time Travel is Dangerous}
      Relating back to the earlier discussion of interactive Behaviours it can be seen
      that there are problems when plugging FRP into the real world: the future is uncertain and
      arbitrary access to the past requires an unbounded level of memory. After much deliberation
      over different compromises it was decided that echo would follow a set of rules:
      
      \begin{itemize}
        \item If a Behaviour was last evaluated at time \emph{u} then it can only be evaluated at
        a time \emph{t} if \emph{t} $\geq$ \emph{u}.
        \item An Event holds the last time it occurred and the time up to which it hasn't occurred
        again (the time it is `valid' until).
        \item An Event is valid up until the time equivalent to `now' (the actual current time).
        \item If a Behaviour depends on some Event then we can only evaluate the Behaviour up to
        the time the Event is valid.
      \end{itemize}
      
      With this set of rules in place, it becomes clear that we will only ever use a fixed amount of memory
      for an Event and that a Behaviour can not be evaluated in the past or the future. Will
      introduce a new operation on a \emph{Behaviour[T]} for this:

\begin{verbatim}
eval() : T
beh.eval() = beh.at(now())
\end{verbatim}       

      We will also remove the `at' function on Behaviours from the public API (it will still be
      use within the framework). 
      
      Of course, the rules can still be broken if we introduce multi-threading and this is best illustrated
      with an example:
      
\begin{verbatim}
val beh = new Behaviour(time => 5)

class Runner extends Runnable() {
  def run() {
    beh.eval()
  }
}

new Thread(new Runner).start()
new Thread(new Runner).start()
\end{verbatim}       

      Here we have two threads continuously evaluating one Behaviour. This means that one thread
      might call \emph{eval()} and then be scheduled out allowing the other to proceed to evaluate
      at a later time. If the original is then scheduled back in it is possible it will
      be evaluating the Behaviour at a time less than the time second evaluated it at. This suggests
      that we desire the \emph{eval} operation to be `atomic'. This can be achieved easily for standalone
      Behaviours using standard locking tools in Scala.
      
      We run into further problems however if the Behaviour is more complex. Suppose, in the above example,
      the Behaviour was defined as follows:

\begin{verbatim}
val beh = beh1.until(ev, beh2)
\end{verbatim}        

      Here, \emph{beh} `depends' on \emph{beh1} and \emph{beh2}: whenever we evaluate \emph{beh}
      we also need to evaluate \emph{beh1} or \emph{beh2}. This shows that Behaviours can't just
      be locked on a standalone basis but must share a lock with all their dependencies and other Behaviours
      that share those dependencies. By this it is meant that a group of dependent Behaviours can
      only be read by one thread at a time.
      
      There was one further problem presented due this restriction and that was concerned with
      implementing the `snapshot' operation. For instance, imagine we execute the following code
      at time \emph{t}:
      
\begin{verbatim}
val ev = beh.snapshot(someEvent)
\end{verbatim}

      This is fine if \emph{someEvent} has not occurred at \emph{t}. If it has however at some time
      before \emph{t} then in whatever way this is implemented we are forced to evaluate \emph{beh}
      at a time in the past. If \emph{beh} depends on any Events this time might be before their
      last occurrence and so we would effectively be returning an incorrect result. Dealing with this
      presented an interesting problem that caused quite a radical change. It was decided that FRP
      code (FRP operations on FRP types) could only be performed in an initial setup phase of a program.
      This setup phase would happen `instantly' as the internal notion of time would be frozen and Event
      occurrences would not be processed by the framework. As Scala programs are defined inside a singleton
      object it was decided that a user would simply be able to extend a class with this object and then
      write their FRP code in a `setup' function:
 
\begin{verbatim}
object App extends EchoApp {
  def setup(args : Array[String]) {
    // FRP code here
  }
}
\end{verbatim}

      This allows the FRP world to be setup in one distinct stage. It should be noted that
      any functions `setup' calls would also be able to run FRP code. Only FRP code that runs
      after the `setup' function has finished executing will cause an error.
      
    \subsection{Time Freezing}
      Some FRP operations require some internal action to happen `exactly' when an Event occurs. For instance,
      the snapshot operations needs to be able to sample a Behaviour at the same time as an Event occurs.
      We can see this will be a problem as time conceptually moves forward as the code to
      occur an Event is executing. To deal with this in echo we treat occurrences as if they happen instantly. This
      creates a similar effect to the initial setup phase as time is frozen at its current value for the duration
      of the operation and other Events must wait to occur. This may sound like it would cause large lags in time
      but any large lags would only be nanoseconds in length and it allows the very useful `snapshot' operation
      to be implemented without sacrificing its original denotational semantics.
    
    \subsection{Function Renaming}
      Although a small change it should also be noted that one of the functions originally defined
      has be renamed for the echo framework - `snapshot' was renamed to `sample'. This was done as
      it was felt it added more clarity to the function name was more appropriate given its use.
      
    \subsection{Function Additions}
      When considering different applications of the framework and also in an attempt to
      tackle the changes in expressiveness between Haskell and Scala some operations
      were added to the FRP in echo.
      
      Lifting of constants has already been discussed for echo but lifting of operators and
      functions presents a few problems. This is because function in Scala (as discussed earlier)
      are usually attached to classes. For instance, there is no `'+' operator in Scala: the numeric
      types have a plus function defined in their class declarations (how Scala actually performs additions
      is a topic for discussion elsewhere). This means that we cannot `lift' many of the operations we
      would want to. To solve this problem we can use a series of map functions. We can define these
      here for a \emph{Behaviour[T]}:

\begin{verbatim}
map[U](func : T => U) : Behaviour[U]
beh.map(func).at(t) = func(beh.at(t))

map2[U, V](beh1 : U)(func : (T, U) => V) : Behaviour[V]
beh.map2(beh1)(func).at(t) = func(beh.at(t), beh1.at(t))
\end{verbatim}        

      These functions allow Behaviours to be transformed and combined using functions
      normally applied to static values.
      
      In Hudak and Elliott's original paper an example is constructed to create a Behaviour
      that toggles back and forth between two given Behaviours whenever an Event occurs. This
      example took advantage of Haskell's lazy evaluation. While Scala is able to use lazy evaluation
      it requires extra work and at the time of writing requires some complex syntax. To tackle this
      a `toggle' function was added to echo to provide the functionality in the example out of the box.
      This is defined here for a \emph{Behaviour[T]}:
      
\begin{verbatim}
toggle(ev : Event[T], beh1 : Behaviour[T]) : Behaviour[T]
\end{verbatim}        
      
      It was felt that another common use of combining Behaviours and Events would be
      to switch from one Behaviour to another when an Event occurs (but not back again):

\begin{verbatim}
until(ev : Event[T], beh1 : Behaviour[T]) : Behaviour[T]
beh.until(ev, beh1).at(t) = new Switcher(beh, event.map((t,v) => beh1)).at(t)
\end{verbatim}        
      
      It is also useful to be able to to filter Events to produce a new one that only
      contains occurrences matching a predicate. This allows a Behaviour to only react
      to some occurrences from an Event. This is defined on an \emph{Event[T]}:

\begin{verbatim}
filter(predicate : T => Boolean) : Event[T]
event.filter(predFunc).occs = event.occs.filter(predFunc)
\end{verbatim}        
      
  \section{Correctness of Implementation}
    Of course, there needs to be some way of verifying that the implementation is correct. By
    this it is meant that it follows both the semantic rules set forth by the original FRP specification
    and also the rules and constraints laid out in this report.
    
    This achieved in echo using formal program testing. More specifically, Behaviour Driven
    testing was carried out throughout the development process using the Scala framework `Specs'
    to verify that implementations were correct. `Specs' and BDT involve series of tests for
    separate program `invariants' or `behaviours'. For example to test the Behaviour `map' function
    in `Specs' we can do the following:

\begin{verbatim}
"provide a map function" >> {
"returning a Behavior thats value is func(this.at(t))" in {
  val beh = new Behavior(time => 5)
  val func: Int => String = {
    int => int.toString
  }

  beh.map(func).eval() mustEqual "5"
}
}
\end{verbatim}      

  These tests were generally written before implementing a feature and then run after every compilation. 
  This not only allowed the original implementation's correctness to be verified but
  allowed refactoring to be carried out very quickly and easily as testing would verify that any changes had not
  made the implementation incorrect.