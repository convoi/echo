\chapter{Implementation}
  Throughout its history, FRP implementations have generally been carried out in Haskell. Due to the movement
  in setting and also in an effort to create a efficient implementation without sacrificing FRP's expressive power
  several design and implementation choices were made for Echo. Choices were also made
  to provide to provide programmers with simpler experiences and to hopefully allow Echo to be
  as powerful as possible.
  
  \section{Scala}
    Scala is a multi-paradigm programming language developed to run on the Java Virtual Machine \cite{Odersky2004}. Scala 
    programs are
    generally structured in an object oriented programming style like Java but it also incorporates many features of
    functional languages such as closures, high order functions, lazy evaluation, immutable state and currying. These 
    features 
    aid
    implementation of FRP concepts as closures will allow for Behaviour definition. Also, functions such as \emph{map} on
    Behaviours and Events to be expressed in a more natural and succinct form in a functional language. It is
    important to note that although Scala is thought of as `functional' its functions are not `pure'. This means
    that object functions and closures can cause side effects and can be affected by external states.

    Scala also possesses an advanced, powerful and feature rich static type system. This provides large
    advantages when implementing FRP. Firstly, Scala supports generic classes and it can be quite easily seen
    that both Behaviours and Events will rely on this feature for type safety. Scala additionally provides
    a rather unique type conversion functionality that allows types to be converted on the fly. This provides
    incredibly expressive and powerful ways to embed FRP into Scala and will be explored later.
    
    One slight problem with implementing FRP in Scala is that, unlike Haskell, it is an `impure'
    language. In terms of functional programming, `purity' refers to a languages ability to write functions
    thats results are affected by some external estate or in turn effect some external state. For instance,
    this means that a function that defines a Behaviour could be impure and could therefore cause side effects
    whenever the Behaviour was evaluated. There has been some work into pure function verification in Scala but it has not materialised into an  
    implementation \cite{Nordenberg}. Due to this, we will only deal with Behaviours defined using pure Behaviours in Echo.

  \section{Flat Time}
    Time in Echo is implemented as a flat value type defined exactly as it is in 
    section 2 of this report (as an alias of Double). The internal implementation does take advantage
    of general assumptions about Time. For instance, the \emph{merge} function for Events takes advantage
    of the fact that an Event without occurrences still has a has a lower bound of when it can occur (again described
    in section 2). 
    
    Additionally a function is defined in Echo to return the current time:
    
\begin{verbatim}
def now() : Time = System.nanoTime() - startTime
\end{verbatim}      

    Here \emph{startTime} refers to the time the current Echo program started executing and this is
    set by the framework at run time. It should be noted that \emph{nanoTime} (returning current time in
    nano seconds) is the most accurate value available for the current time in Scala. However due the restrictions
    of Java Virtual Machine implementations of this function will start to return incorrect results 
    after running for around 292 years due to an overflow in the calculation (thought to be an allowable
    problem).
    
  \section{Object Orientation}
    Although it may seem obvious it is worth noting that both Behaviours and Events are implemented in
    Echo using Scala's object oriented definitions. For instance, a Behaviour[T] is simply
    implemented as a generic class with a Time =$>$ T function parameter. An example Behaviour
    could be instantiated as follows:

\begin{verbatim}
val beh = new Behaviour[Double](time => sin(time))
\end{verbatim}  

    A discussion of Event's implementation will feature later.
    
    It would be possible to implement the original FRP operations as standalone functions. However, so
    as to conform to Scala's object oriented style the operations will be attached to their respective
    classes. For instance, the \emph{merge} operation will be used in the following way in Echo:
    
\begin{verbatim}
val merged = event1.merge(event2)
\end{verbatim}

  This allows the FRP code to fit more naturally into Scala. We will also replace the \emph{switcher} function
  with a type that extends Behaviour rather than a function:
  
\begin{verbatim}
val switch = new Switcher(iniBeh, behEvent)
\end{verbatim}

  \section{Implicit Lifting of Constants}
    As mentioned earlier, one of the most powerful features of Scala is it's ability to convert between
    types at run time. This is facilitated by the \emph{implicit} key word allowing a function to be defined
    that is automatically executed when a corresponding type mismatch occurs. This will be used in Echo to allow 
    static values to be converted to Behaviours \emph{implicitly}.. This is implemented in the following way:  

\begin{verbatim}
implicit def liftConst[T](value : T) : Behaviour[T] = {
  new Behaviour(time => value)
}
\end{verbatim}              
    
    This would allow the following for instance:

\begin{verbatim}
val beh : Behaviour[Int] = 0
val switcher = new Switcher(false, eventBeh)
\end{verbatim}       

    It can immediately be seen that this allows the programmer to switch between static and more complex
    Behaviours in a very naturalistic manor.
    
    Another advantage of specifying constant Behaviours this way is we can take advantage of the fact that
    we know the Behaviours value will not need to be recalculated. We do this by returning a Behaviour
    that simply stores the static value rather than a function to compute it. This means that whenever the Behaviour
    is evaluated it simply returns a value rather than executing a function.
    
  \section{Extendable Events}
    FRP Events provide a very useful abstraction for external stimuli such as I/O and user input. Of course
    to allow for this and also so that Events can be useful in anyway they must be allowed to actually `occur'.
    It is noticeable that the original denotational semantics \cite{Elliott1997} do not explore how to achieve
    this and it is assumed that this is because the actual occurring of Events is really just an implementation
    issue. As discussed earlier (section 3), previous implementations of Events in Scala add an operation to Event's
    public API to allow it to occur. Here we will show an alternative approach.
    
    When observing how Events are composed (using \emph{merge} and \emph{map} operations) it can be seen that
    only a subset of Events actually `occur' while others are simply views and combinations of these.
    In `echo' this is modelled directly. FRP Events are represented by two traits: Event[T] and EventSource[T].
    Event has concrete implementations of \emph{map} and \emph{merge} operations that
    return concrete Event[T] instantiations. EventSource[T] extends
    from Event[T] and additionally has a protected level concrete function `occur':
    
\begin{verbatim}
occur(value : T) : Unit
\end{verbatim}  

    This function causes the EventSource[T] to `occur' with the given value (with the time the call was made). This allows 
    programmers
    to easily build FRP Event abstractions for any form of input without exposing the \emph{occur} action itself.
    As an example, Here is a simple implementation for an Event that occurs once with the value `5' when it 
    is created:
    
\begin{verbatim}
class Five extends EventSource[Int] {
  occur(5)
}
\end{verbatim}

    It should also be noted that the \emph{occs} function on Events is not actually available publicly in Echo. This
    choice was made because an Event does not retain all its Events so anything returned from this 
    function would be incorrect and accessing a Event's past data does not really
    provide any useful functionality.   

  \section{Time Travel is Dangerous}
    Relating back to the discussion of interactive Behaviours in section 2 it can be seen
    that there are problems when plugging FRP into the real world: the future is uncertain and
    arbitrary access to the past requires an unbounded level of memory. After much deliberation
    over different compromises it was decided that Echo would follow a set of rules:
    
    \begin{itemize}
      \item If a Behaviour was last evaluated at time \emph{u} then it can only be evaluated at
      a time \emph{t} if \emph{t} $\geq$ \emph{u}.
      \item An Event holds the last time it occurred and the time up to which we know it hasn't occurred
      again (the time it is `valid' until).
      \item An Event is valid up until the time equivalent to `now' (the actual current time).
      \item If a Behaviour depends on some Event then we can only evaluate the Behaviour up to
      the time the Event is valid. We additionally can only evaluate it at a time greater than
      or equal to the Event's last occurrence.
    \end{itemize}
    
    These rules restrict Behaviours to what we will refer to as `real-time' evaluation as the times
    we evaluate a Behaviour at become monotonic in nature. Additionally, as Event occurrences will
    be happening in real time we will see that the lower bound that we can evaluate a Behaviour at will
    constantly be pushed up to the present (leaving only `now' as a possible evaluation time). Although
    other times will be available for evaluation at times (there will often be a gap between `now' and
    the last Event occurrence) it would be monotonous to expose these bounds to the programmer so
    in Echo explicit evaluation of a Behaviour is entirely removed from the public API (the
    `at' operation) and replaced with a real-time evaluation function:

\begin{verbatim}
eval() : T
beh.eval() = beh.at(now())
\end{verbatim}  

    This restriction may seem like a significant loss of expressivity. However as we wish to be able to treat
    Behaviours and Events as first class values we would actually like to avoid ever explicitly evaluating
    them: this would ideally be carried out by abstractions for graphics or other kinds of output. For instance
    if we wanted to set some colour in a user interface equal to a Behaviour then explicitly evaluating the Behaviour
    periodically and setting the colour using a static value would defeat the point of using FRP. We will see in later
    discussion that we can easily create abstractions to hide this type of logic.
    
    It should also be noted that restricting evaluation times for Behaviours forces us
    to remove the \emph{timeTransform} operation for Behaviours (as this allows programmers to explicitly
    evaluate Behaviours in the past or future).
  
    \section{Setup Phase}
      There was one further problem presented due the restriction on Behaviour evaluation and that was concerned with
      implementing the \emph{snapshot} operation. For instance, imagine we execute the following code
      at time \emph{t}:

  \begin{verbatim}
  val ev = beh.snapshot(someEvent)
  \end{verbatim}

      This is fine if \emph{someEvent} has not occurred at \emph{t}. If it has occurred however at some time
      before \emph{t} then in whatever way this is implemented we are forced to evaluate \emph{beh}
      at a time in the past. If \emph{beh} depends on any Events this time might be before the Event's
      last occurrence and so we would effectively be returning an incorrect result. Dealing with this
      presented an interesting problem. It was decided that FRP
      code (FRP operations on FRP types) could only be performed in an initial setup phase of a program.
      This setup phase would happen `instantly' as the internal notion of time would be frozen and Event
      occurrences would not be processed by the framework. As Scala programs are defined inside a singleton
      object it was decided that a user would simply be able to extend a class with this object and then
      write their FRP code in a \emph{setup} function:

  \begin{verbatim}
  object App extends EchoApp {
    def setup(args : Array[String]) {
      // FRP code here
    }
  }
  \end{verbatim}
  
    This allows the FRP world to be setup in one distinct stage. It should be noted that
    any functions \emph{setup} calls would also be able to run FRP code. Only FRP code that runs
    after the \emph{setup} function has finished executing will cause an error.
    
    It should be noted that isolated FRP objects can still be instantiated after the setup phase. This is
    to allow for the possibility Event[Behaviour[T]] or Event[Event[T]] objects.
  
  \section{Constant Events}
    It useful to be able to define an Event that occurs only once. This is provided in Echo
    via the Event companion apply function:

\begin{verbatim}
val const = Event("Once")
\end{verbatim}    

    This would create an Event with a single occurrence of the value ``Once'' at time \emph{0} (as
    it occurs at the end of the setup phase).
    Although it may not be as useful it should also be possible to instantiate a never occurring
    Event. This can be achieved in a similar manner:

\begin{verbatim}
val empty = Event()
\end{verbatim}
  
  \section{Atomic Input \& Output}
    
     By this point it is noticeable that the FRP world that Echo builds only really has one form of input and one
     form of output action: \emph{occur} and \emph{eval} respectively. Conceptually we think of both of these
     actions happening instantly. For instance, when a Behaviour is sampled it will need to do some evaluation
     and during that evaluation nothing in the FRP world should change. This suggests that the \emph{occur}
     and \emph{eval} actions should both be atomic with respect to this world so an Event cannot occur during
     a Behaviour evaluation and vice versa. 
   
     This concept is implemented in Echo using a locking object for the FRP world. If a Behaviour is evaluated it
     must first acquire this lock and the same goes for the Event \emph{occur} action. This implementation
     also provides a very large performance gain: Event occurrence times and evaluation times now form one
     monotonically increasing timed chain \footnote{We can actually model evaluation and occurrences of an Event as an 
     Event!} of actions. This fact means that we only ever need to store one occurrence for each Event: if an Event occurs
     at time \emph{t} it will not be evaluated until a time greater than or equal to \emph{t}. This allows
     Events to only use a bounded level of memory.
   
     It could be argued that this implementation is problematic as there may be some lag between a real world
     occurrence and it being processed by the FRP system in Echo. For instance, if a user clicks on a button
     while some network Event is occurring the button click will have to wait to `occur' in the FRP system.
     However, if these actions weren't atomic there would be the possibility of latency inside the FRP
     system itself. Taking the example above, imagine the two real world occurrences are delivered on two
     separate threads and that the button click arrives at time \emph{t} and the network occurrence
     arrives later at time \emph{u}. In this
     situation our FRP system's accuracy is at the mercy of thread scheduling as the button click
     could be part of the way through the \emph{occur} action and then be scheduled out to allow the network
     occurrence to happen. If these Events had been merged we would effectively skip a state:
     we would never observe the occurrence at \emph{t}. It can be seen here that this implementation
     choice is required so we are able to deal properly with Event's generated from multiple sources. 
     \footnote{It is also required if Behaviours are evaluated concurrently as thread scheduling could
     allow a Behaviour to be evaluated at a time later than the present if an evaluating thread is
     scheduled out for some time.}
     
  \section{Function Renaming}
    Although a small change it should also be noted that one of the functions originally defined
    has be renamed for the Echo framework - \emph{snapshot} was renamed to \emph{sample}. This was done as
    it was felt it added more clarity to the function name was more appropriate given its use.
    
  \section{Function Additions}
    When considering different applications of the framework and also in an attempt to
    tackle the changes in expressiveness between Haskell and Scala some operations
    were added to the FRP in Echo.
    
    Lifting of constants has already been discussed for Echo but lifting of operators and
    functions presents a few problems. This is because functions in Scala (as discussed earlier)
    are usually attached to classes. For instance, there is no `+' operator in Scala: the numeric
    types have a \emph{+} function defined in their class declarations (how Scala actually performs additions
    is a topic for discussion elsewhere). This means that we cannot `lift' many of the operations we
    would want to. To solve this problem we can use a series of map functions. We can define these
    here as part of Behaviour[T]:

\begin{verbatim}
map[U](func : T => U) : Behaviour[U]
beh.map(func).at(t) = func(beh.at(t))

map2[U, V](beh1 : U)(func : (T, U) => V) : Behaviour[V]
beh.map2(beh1)(func).at(t) = func(beh.at(t), beh1.at(t))
\end{verbatim}        

    These functions allow Behaviours to be transformed and combined using functions
    normally applied to static values.
    
    In Hudak and Elliott's original paper an example is constructed to create a Behaviour
    that toggles back and forth between two given Behaviours whenever an Event occurs. This
    example took advantage of Haskell's lazy evaluation. While Scala is able to use lazy evaluation
    it requires extra work and at the time of writing requires some complex syntax. To tackle this
    a \emph{toggle} function was added to Echo to provide the functionality in the example out of the box.
    This is defined as part of Behaviour[T]:
    
\begin{verbatim}
toggle(ev : Event[T], beh1 : Behaviour[T]) : Behaviour[T]
\end{verbatim}        
    
    It was felt that another common use of combining Behaviours and Events would be
    to switch from one Behaviour to another when an Event occurs (but not back again):

\begin{verbatim}
until(ev : Event[T], beh1 : Behaviour[T]) : Behaviour[T]
beh.until(ev, beh1).at(t) = new Switcher(beh, event.map((t,v) => beh1)).at(t)
\end{verbatim}        
    
    It is also useful to be able to to filter Events to produce a new one that only
    contains occurrences matching a predicate. This allows a Behaviour to only react
    to some occurrences from an Event. This is defined as part of Event[T]:

\begin{verbatim}
filter(predicate : T => Boolean) : Event[T]
event.filter(predFunc).occs = event.occs.filter(predFunc)
\end{verbatim}  

  \section{Behaviour Caching}
    Behaviours possess quite a major inefficiency in the form of redundant evaluation:

\begin{verbatim}
beh.at(t)
beh.at(t)
\end{verbatim}  

    Here \emph{beh} is evaluated at \emph{t} twice. It can be seen that we could simply cache the last
    computed value so we could return this value instantly if the Behaviour was sampled at that time again.
    We take this approach internally in Echo so as to avoid redundantly recomputing values. 
  
  \section{Test Driven Development}
    Echo was mainly developed using Test Driven Development \cite{Cunningham}. This methodology is based around the idea
    of specifying some program feature with a test and then writing code to pass that test. The hope
    is that doing so will not just ensure testing is thorough (as all features will require tests) but
    that solutions are not over thought (the code is as simple as possible). TDD also encourages
    refactoring as if refactored code passes the tests for the initial solution we can be more confident
    that it is semantically equivalent in it's execution.

    Test implementation for Echo was carried out using the Scala framework `Specs' \cite{EricTorreborre}. This framework
    encompasses a DSL that allows tests to easily be defined. For instance, this code tests the implicit lifting
    of constants feature:

\begin{verbatim}
"Echo" should {
  "allow values to lifted to constant Behaviours" in {
    val beh: Behaviour[Int] = 5
    beh.eval() mustEqual 5
  }
}
\end{verbatim}

    Carrying out testing such as this was imperative to the project goals as it was very important to be able
    to verify that code did not only compile and run but that implementations matched the original denotational 
    semantics. In other words developing Echo with TDD allowed for verification that the implementation
    was correct throughout the development process.