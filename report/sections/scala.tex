\chapter{Scala}
  Scala is a multi-paradigm programming language developed to run on the Java Virtual Machine \cite{Odersky2004}. Scala 
  programs are
  generally structured in an object oriented programming style like Java but it also incorporates many features of
  functional languages such as closures, high order functions, lazy initialisation and currying. These features aid
  implementation of FRP concepts as closures will allow for Behaviour definition and functions such as map on
  Behaviours and Events to be expressed in a more natural and succinct form. It is
  important to note that although Scala is thought of as `functional' its functions are not `pure'. This means
  that object functions and closures can cause side effects and be altered by external states.

  Scala also possesses a advanced, powerful and feature rich static type system. A few of these provide large
  advantages when implementing FRP. Firstly, Scala supports generic classes and it can be quite easily seen
  that both Behaviours and Events will rely on this feature for type safety. Scala additionally provides
  a rather unique type conversion functionality that allows types to be converted on the fly. This provides
  incredibly expressive and powerful ways to embed FRP into Scala and will be explored later.
  
  \section{Problems}
    One quite significant problem with implementing FRP in Scala is that, unlike Haskell, it is an `impure'
    language. In terms of functional programming, `purity' refers to a languages ability to write functions
    thats results are affected by some external estate or in turn effect some external state. For instance,
    this means that a function that defines a Behaviour could in fact cause side effects meaning that
    every time the Behaviour was evaluated it would execute this side effect. There has been some work into
    pure function verification in Scala and this may be useful for implementing FRP. However, enforcing
    purity in Scala can cause serious problems due to high level library code that is executed for
    what might seem like self contained systems.
  
  \section{Previous Work}
    As one might imagine, there has been previous work in implementing FRP in Scala. 
    
    \subsection{Reactive}
      Arguably the most significant of FRP work in Scala is the library `reactive' created by `nafg'. This framework was 
      designed to allow for the creation of functional reactive web applications however here we are concerned with
      the basic FRP library `reactive-core'. Unfortunately while claiming to be an FRP library `Reactive'
      has a couple of problems.
      
      The framework is based around two types designed to represent Behaviours and Events respectively: Signals
      and EventStreams. EventStreams are very close to the original Event definition but lack the
      useful `merge' operation however it is with Signal that the core differences lie. Signal is defined as follows:
      
      \begin{quote}
        In practical terms, a Signal has a current value, and an EventStream that, whenever the Signal's value changes,     
        fires the new value.
      \end{quote}  
      
      Although it is claimed that this results in Signals having a continuous value we can see
      that at any point in time it is really just a static value that changes at discrete point in
      time. We can see that this doesn't allow us to represent purely continuous values. For instance
      how would we represent the following basic Behaviour with a Signal:
      
\begin{verbatim}
new Behaviour(time => time)
\end{verbatim}

      It can be seen that a Signal[T] can however be represented in terms
      of the `switcher' function meaning that it does have some relevance to Behaviours:
      
\begin{verbatim}
def signal[T](v : T, ev : Event[T]) = {
switcher(lift(v), map(ev, o => lift(o)))
}
\end{verbatim}  

      This functionality was additionally encapsulated within the `stepper' function in Conal Elliott's
      `Push-Pull' paper:

\begin{verbatim}
stepper[T](v : T, ev : Event[T]) : Behaviour[T]
stepper(v, ev) = switcher(lift(v), map(ev, o => lift(o)))
\end{verbatim} 

      It can be seen here that although Signal provides a useful programming tool it is not an implementation
      of an FRP Behaviour as it only allows for a subset of a Behaviour's functionality meaning that `Reactive'
      does not successfully implement FRPs semantics.
    
    \section{React}
      While not claiming to be an implementation of FRP the `React' framework proposed in the 
      paper `Deprecating the Observer Pattern' \cite{Maier2010} does utilise a notion of Events that, as cited in the paper,
      has its routes in FRP.
      
      The Event's in React are very similar to the Events in FRAN: they are conceptually infinite
      and have the `merge' and `map' operations defined on them. One slight difference is that React
      Events expose an operation to allow them to `occur' with an `emit' function. This functionality
      is actually never discussed in the original FRP papers and so it should be assumed that this
      functionality should be hidden publicly (it will be shown later how to internalise this action
      effectively). Interestingly, React also contains a notion of Signals, just as Reactive does, however
      it doesn't claim that these are an implementation of Behaviours (although the word `continuous' is
      used loosely in their description). 
      
  It can be seen from these examples that while many of the concepts from FRP are popular
  in the Scala community there has been little attempt to implement FRP within Scala while honouring
  the original denotational semantics.