\chapter{Introduction}

  Dealing with animations, user interactions and I/O are often thought of as monotonous imperative tasks
  in programming. Even in higher level languages programmers often use imperative
  styles within more declarative code. This is somewhat counter intuitive as often many of these tasks are possible to generalise: adding
  some action to a button click or looping through some animation for instance. Functional Reactive Programming
  is a programming paradigm that aims to provide programmers with the tools to create high level,
  declarative abstractions of these kinds of tasks. This report is concerned with a project to
  implement Functional Reactive Programming as a framework dubbed `Echo' \cite{Stott} for the multi-paradigm programming language
  Scala.

  Past implementations of FRP have had problems for various reasons: many have had to sacrifice the original semantics or functionality in some way to avoid
  memory leaks (often referred to as ``time leaks'') or other inefficiencies. One of the goals
  of this project was to try and implement FRP in a way that allowed for a controlled usage of memory while also
  adhering to the original denotational semantics. 
  
  FRP
  has also commonly been implemented in Haskell. Implementing FRP in a more industrial language such as Scala allows
  further exploration of the paradigm in a different setting. Additionally, Scala's portability and interoperability with
  languages such as Java, JRuby, Closure and and Groovy should hopefully allow for easy wide spread use across many platforms and languages. 

  This report will first deal with an introduction to the formal semantics of FRP. There will also be a discussion of previous    
  work carried out using FRP and Scala.
  Finally, an exploration of the implementation of Echo itself and its surrounding work will be carried out with an
  evaluation of its current functionality and adherence to the project goals.