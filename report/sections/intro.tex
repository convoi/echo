\chapter{Introduction}

Dealing with animations, user interactions and I/O are often thought of as monotonous imperative tasks
in programming. This is somewhat counter intuitive and often forces programmers to use imperative
styles within more declarative code. Many of these tasks are possible to generalise in fact: adding
some action to a button click, looping through some animation. Functional Reactive Programming
is a programming paradigm that aims to provide programmers with the tools to create high level,
declarative abstractions of these kinds of tasks. This report is concerned with a project to
implement Functional Reactive Programming as a framework dubbed `echo' for the multi-paradigm programming language
Scala.

For various reasons, implementations of FRP have had various problems in the past: many have have problems
with efficiency or had to sacrifice original semantics or functionality is some way. One of the goals
of this project was to try and implement FRP in way that avoid these problems as much as possible. FRP
has also commonly been implemented in Haskell. Implementing FRP in a more industrial language such as Scala allows
further exploration of the paradigm in a different setting. Scala's portability and interoperability with
languages such as Java, JRuby, Closure and and Groovy should allow for easy wide spread use across many platforms and languages. 

This report will first deal with an introduction to the formal semantics of FRP and current research in that area.
There will also be a discussion of Scala as a choice for this project and previous work carried out using FRP and Scala.
Finally, an exploration of the implementation of `echo' itself and its surrounding work will be carried out with an
evaluation of its current functionality and adherence to the project goals.