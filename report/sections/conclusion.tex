\chapter{Conclusion}

A successful implementation of the FRP paradigm has been shown to be accomplishable. This implementation
managed to adhere to the original semantic while not compromising performance (although some functionality was
lost). Additionally `echo' also showed promise in actual use as building interactive application with it
was a relatively simplistic process compared o standard comparative techniques. Programs do require abstractions
to built for input/output systems and external libraries but these abstractions, if built properly, are highly
reusable.

Scala's features in general were more beneficial than detrimental when implementing the framework. However,
impure Behaviours still do pose some problematic characteristics although the restrictions made on 
`echo' (mainly real-time evaluation) circumvent these to some degree.

The exploration of FRP throughout the project has shown it can in fact be a very useful approach to programming,
even in areas that hadn't been touched on by previous works such as networking. It is limited by
both its learning curve and its need for pre-built abstraction but with promotion and work on frameworks
and abstractions for mainstream languages it could become a common feature among various areas of programming.