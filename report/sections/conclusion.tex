\chapter{Conclusion}

A successful implementation of the FRP paradigm in Scala has been shown to be accomplishable. This implementation
managed to adhere to the original semantic while allowing for bounded use of memory (although some functionality was
lost). Additionally Echo also showed promise in actual use as building interactive application with it
was a relatively simple process compared to standard imperative techniques. Programs do require abstractions
to be built for input/output systems and external libraries but these abstractions, if built properly, are highly
reusable.

In general, Scala's features were more beneficial than detrimental when implementing the framework. However,
impure Behaviours still do pose some problematic characteristics. Implementing some way of defining pure functions
in Scala would be a very worthwhile piece of work for the future so this problem could be avoided.

Other future work that could be carried out on Echo itself would mainly include expanding the
UI framework to allow users to create many different types of graphical applications. This could include
create graphing components, menu systems and many more. Additionally,
work could be carried out on an improved locking system described in section~\ref{sec:eff}. There is also much
work that could be carried out with Echo. For instance, the Sender and Receiver types could be utilised
to create web servers or distributed systems with enough work. Experimenting with FRP in domains such as
these could provide a fresh perspective on both the paradigm and the domains themselves and would be a very
worthwhile endeavour.