\chapter{Conclusion}

A successful implementation of the FRP paradigm in Scala has been shown to be accomplishable. This implementation
managed to adhere to the original semantic while allowing for bounded use of memory (although some functionality was
lost). Additionally Echo also showed promise in actual use as building interactive application with it
was a relatively simple process compared to standard imperative techniques. Programs do require abstractions
to be built for input/output systems and external libraries but these abstractions, if built properly, are highly
reusable.

Scala's features in general were more beneficial than detrimental when implementing the framework. However,
impure Behaviours still do pose some problematic characteristics. Implementing some way of defining pure functions
in Scala would be a very worthwhile piece of work for the future so this problem could be avoided.

Other possible future work that could be carried out might involving improvements and extensions of the current UI
and I/O frameworks. Abstractions over additional areas of programming would most likely also be useful. Additionally,
work could be carried out on an improved locking system described in chapter 6.