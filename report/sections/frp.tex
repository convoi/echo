\chapter{Functional Reactive Programming}
  \section{Origins}
  \label{sec:origins}
    Functional Reactive Programming was originally proposed in the 1997 paper `Functional
    Reactive Animation' by Conal Elliott and Paul Hudak \cite{Elliott1997}. In this paper Elliott and Hudak
    set forth a ``collection of data types and functions'' designed for creating interactive 
    animations in the form of a framework for Haskell named `Fran'. The paper eventually went on to
    win an award distinguishing it as the most influential paper of ICFP97 \cite{SIGPLAN}. 
    
    Since the original proposal of the paradigm there has been a great amount of research
    into implementation and modification of FRP. Hudak has mainly focused on experimenting with various
    implementation methods and a modification to the original semantics categorised as ``Arrow FRP''. Elliot's
    work has dealt with FRP on a more conceptual level. This work culminated in a further significant paper
    ``Push-Pull FRP'' \cite{Elliott2009}. In this paper Elliott attempted to clarify much of the semantics in a way that encouraged more efficient 
    and
    structured implementations. This report will focus on the FRP defined in the original 1997 paper using
    some of Elliott's later work to clarify certain points.

  \section{Semantics}
    \label{sec:semantics}
    Within this section we will explore a selection of the basic denotational semantics defined in Fran
    that we hope to implement in Scala. The semantic definitions will
    be presented using Scala syntax so as to provide continuity throughout the report. Definitions will
    be in the form of a function signature followed by an expression that denotes its operation. This style
    is based on that found in the Fran paper itself.
  
    \subsection{Types}
      The concepts introduced in the Fran paper and in following FRP work are based around two key data types: 
      Behaviours and Events. These are often described in terms of Time so this will also be defined here.
      
      \subsubsection{Time}
        We often think of
        Time as a continuous value and so in in numerical terms this is best modelled by the
        Scala type Double. This is the definition we will use throughout this report:

\begin{verbatim}
type Time = Double
\end{verbatim}    

        There are some interesting properties of Time that we can utilise in FRP. For instance,
        conceptually there is no minimum Double value but with Time we can say that a given Time is `at least'
        some value. For instance, imagine we are recording the times at which a person claps: even though we
        may not know the time of the next clap we can always say that it is greater than `now'. This idea is explored 
        in great detail within two of Conal Elliott's papers \cite{Elliott1997}, \cite{Elliott2009} but will not feature
        heavily in this report for the sake of simplicity.
   
      \subsubsection{Behaviour}
        Behaviours are essentially values that change over time. Semantically we think of them as
        a function from Time to some type T:

\begin{verbatim}
type Behaviour[T] = Time => T
\end{verbatim}
        
        We can access a Behaviour's value at some time \emph{t} with the \emph{at} function:

\begin{verbatim}
def at[T](beh : Behaviour[T], t  : Time) : T
\end{verbatim}
      
      \subsubsection{Event}
        Events can be thought of as an infinite stream of Time-value pairs:

\begin{verbatim}
type Event[T] = Stream[(Time, T)]
\end{verbatim}
        
        As with Behaviours, we define a function to access an Event's data:
        
\begin{verbatim}
def occs[T](ev : Event[T]) : Stream[(Time, T)]
\end{verbatim}

        As this stream is seen as a list of occurrences in Time it is monotonically
        increasing with respect to time: the time of each occurrence is greater than
        or equal to the occurrence's before it.
       
    \subsection{Operations}
      Elliott and Hudak additionally define a set of actions on Behaviours and Events.
      
      \subsubsection{Behaviours}
        We would like functions defined on static values to also be available to Behaviours. Hudak
        and Elliott define a series of functions for accomplishing this:
        
\begin{verbatim}
def lift0[T](t : T) : Behaviour[T]
at(lift0(v), t) = v  
  
def lift1[T, U](func : T => U, beh : Behaviour[T]) : Behaviour[U]
at(lift1(func, beh), t) = func(at(beh, t))

def lift2[T, U, V](func : (T, U) => V, beh1 : Behaviour[T], 
  beh2 : Behaviour[U]) : Behaviour[V]
at(lift2(func, beh1, beh2), t) = func(at(beh1, t), at(beh2, t))
\end{verbatim}

      It can be seen that we can extend this concept to cover functions with any number of input arguments.
      It also may be useful to evaluate Behaviours in a modified time frame. For example, this allows
      slowing down or speeding up a Behaviour. We define this function like so:

\begin{verbatim}
def timeTransform[T](beh : Behaviour[T], 
  time : Behaviour[Time]) : Behaviour[T]
at(timeTransform(beh, timeBeh), t) = at(beh, at(timeBeh, t))
\end{verbatim}
        
      \subsubsection{Events}
        It is useful to be able to merge two Events into one. We can achieve this with the \emph{merge} function:

\begin{verbatim}
def merge[T](ev1 : Event[T], ev2 : Event[T]) : Event[T]
def mergeStr[T](str1 : Stream[Time, T], 
  str1 : Stream[Time, T]) = {
  if (str1.first.time <= str2.first.time) {
    new Stream(str1.first, mergeStr(str1.drop(1), str2))
  } else {
    new Stream(str2.first, mergeStr(str1, str2.drop(1)))
  }
}
occs(merge(ev1, ev2)) = mergeStr(occs(ev1), occs(ev2))
\end{verbatim}            

      It should be noted that if if an occurrence from \emph{ev1} and \emph{ev2} occur `simultaneously' (have the same time) 
      then
      the occurrence from \emph{ev2} will be placed after in the merged stream. Another interesting point to explore
      here is that often we can't evaluate which Event occurred first. Imagine one Event has one occurrence in
      the future but the other has no occurrences: the second still might occur before the first, we just don't know it
      yet. In this situation we can take advantage of the fact that the compared values are values in Time. Assuming
      Events cannot occur in the past we know that the second one will not occur until after the time `now' (the
      present time). Again, this concept is discussed at length by Conal Elliott \cite{Elliott2009}.
       
      Elliott and Hudak's paper also defined a series of transformer functions for Events. Here we define a
     \emph{map} function for allowing the same functionality:

\begin{verbatim}
def map[T, U](ev : Event[T], func : (Time, T) => U)  : Event[U]
occs(map(ev, func)) = occs(event).map(func)
\end{verbatim}     

      A function for flattening Event[Event[T]] to an Event[T] is also defined:

\begin{verbatim}
def delayOccs[T](t : Time, ev : Event[T]) : Event[T]
occs(delayOccs(t, ev)) = occs(ev).map(o => (max(o._1, t), o._2))
\end{verbatim}

\begin{verbatim}
def join[T](ev : Event[Event[T]])  : Event[T]
occs(join(ev)) = occs(map(ev, 
  e => delayOccs(e._1, e._2))).foldLeft(emptEv,
    (e, o) => merge(e, o))
\end{verbatim}        

      One caveat to this function is that any occurrences of Events that occur in \emph{ev} are delayed
      so their occurrence times are at least that of the time that their parent Event occurred.
      
      It should be noted that as the times in Event occurrences are monotonically increasing
      we are able to express the occurrences before some specified time as a finite set. We will
      define a notation for this here:

\begin{verbatim}
def finite[T](ev : Event[T], t : Time) : Stream[(Time, T)]
finite(ev, t) = occs(ev).takeWhile((te, v) => te <= t)
\end{verbatim}
        
      \subsubsection{Reactivity}
        One of the most powerful aspects of FRP is the ability to use Events to punctuate or `switch' Behaviours.
        The original Fran paper defined an \emph{untilB} function that was later renamed (more appropriately in this
        author's mind) to \emph{switcher} \cite{Elliott2009}:

\begin{verbatim}
def switcher[T](ini : Behaviour[T], ev : Event[Behaviour[T]])
at(switcher(ini, ev), t) = at(finite(ev, t).last._2.at(t))
\end{verbatim}

        We can also extend this concept to static values which is very useful in practice:
      
\begin{verbatim}  
def stepper[T](ini : T, ev : Event[T])
stepper(ini, ev) = switcher(lift0(0), map(ev, (t, v) => lift0(v)))    
\end{verbatim}

        Another useful interaction between Behaviours and Events is to be able to sample
        a Behaviour at the exact moment an Event occurs. This functionality was originally defined in the
        form of the \emph{snapshot} operation:

\begin{verbatim}
def snapshot[T, U](beh : Behaviour[T], 
  ev : Event[U]) : Event[(U, T)]
occs(snapshot(beh, ev)) = occs(ev).map((t, v) => (v, at(beh, t)))
\end{verbatim} 

    \subsubsection{Predicate}
      Fran additionally included a function for creating an Event that would mark the first
      time a Behaviour[Boolean]'s value became true after a given time:
      
\begin{verbatim}
def predicate(beh : Behaviour[Boolean], t : Time ) : Event[Unit]
occs(predicate(beh, t)).first = (lowestAfter(beh, t), ())
\end{verbatim}  

      By \emph{lowestAfter(beh, t)} it is meant the lowest value of time after \emph{t} for which 
      \emph{beh}'s value is true. We will see later that this function (as it may appear) is very problematic when
      it comes to implementation.
  
  \section{Subtleties}
    \label{sec:subtle}
    FRP presents some subtle semantic problems and caveats that are not always immediately obvious.
    They will be explored here in an attempt to aid the reader's understanding.
     
    \subsection{Events as Monotonic Streams}
      As stated earlier, we think of Events as streams of Time-value pairs. What isn't immediately
      obvious is that we think of this stream as real-time data: each occurrence is a value that
      `occurs' at the time stated. As stated earlier, this means that the stream is monotonic in
      respect to occurrence times.
      
      With the previous observations in mind it it can be seen that Events should not be treated as a
      standard data structure. Its noticeable that the earlier definition of the \emph{map} operation for
      Events does not provide a way to modify occurrence times. Here we will show why this is
      the case using a modified version of this function.
      
      First we define the new function:
      
\begin{verbatim}
def dangerMap[T, U](ev : Event[T], 
  func : (Time, T) => (Time, U))  : Event[U]
occs(map(ev, func)) = occs(event).map(func)
\end{verbatim}

      Now we can do the following:
      
\begin{verbatim}
val event2 = dangerMap(event1, (t, v) => (t - 1, v))
\end{verbatim}

      \emph{event2} makes no sense as when there is a new occurrence in \emph{event1} at time 
      \emph{t} a corresponding occurrence will be generated for \emph{event2} but at time 
      \emph{t - 1}. This means occurrences of this Event are happening in the past! We must
      be careful to avoid allowing situations like these to happen when implementing FRP.
      
    \subsection{Events are Infinite}  
      It is also important to highlight that an Event is conceptually \emph{infinite}. This means
      that they must be evaluated lazily: we can't merge two infinite streams when we actually
      call \emph{merge} as it would never finish! This also means that there are some things we
      can't express for an Event such as its `length' or its `last' element.
    
    \subsection{Interactive Behaviour}
      FRP runs into some of its most significant problems when it is combined with any
      sort of interactivity - unfortunately an area of programming it initially seems
      well equipped for. This is best explained with an example. For this we will use an 
      Event[(Int, Int)] \emph{mousePos} that occurs every time the user's mouse moves with the new position:

\begin{verbatim}
val pos = switcher(lift0((0, 0)), 
  map(mousePos, (t, pos) => lift0(t, pos)))
at(pos, 2)
at(pos, 5)
\end{verbatim}    

      Now, suppose the time when this code executes is \emph{4}. We can see here
      that we are going to be evaluating \emph{pos} at a time greater than `now'. This is a problem as we don't
      actually know if or when the mouse will move in the future. To deal with this we could delay the return
      result until we know the answer, raise an error or modify the original semantics. However, none of these solutions are
      particularly useful.
      
      What about the past? Evaluating \emph{pos} at times before `now' certainly makes sense. However, if arbitrary
      access is allowed to the past in this way then we would have to store every occurrence
      of an Event. It may be the case that this is how Events must be implemented but it does
      mean that any FRP program would use an increasing amount of memory as its run time increased (not a 
      desirable property of any piece of code). This will also result in very large Event data sets which
      may be impossible to traverse quickly for performing operations such as \emph{at} in real-time.
      
      These examples shows that implementing FRP without some restriction might either involve modifying the
      semantics or sacrificing efficiency.